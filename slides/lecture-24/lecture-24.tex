\documentclass{beamer}\usepackage[]{graphicx}\usepackage[]{color}
%% maxwidth is the original width if it is less than linewidth
%% otherwise use linewidth (to make sure the graphics do not exceed the margin)
\makeatletter
\def\maxwidth{ %
  \ifdim\Gin@nat@width>\linewidth
    \linewidth
  \else
    \Gin@nat@width
  \fi
}
\makeatother

\definecolor{fgcolor}{rgb}{1, 0.894, 0.769}
\newcommand{\hlnum}[1]{\textcolor[rgb]{0.824,0.412,0.118}{#1}}%
\newcommand{\hlstr}[1]{\textcolor[rgb]{1,0.894,0.71}{#1}}%
\newcommand{\hlcom}[1]{\textcolor[rgb]{0.824,0.706,0.549}{#1}}%
\newcommand{\hlopt}[1]{\textcolor[rgb]{1,0.894,0.769}{#1}}%
\newcommand{\hlstd}[1]{\textcolor[rgb]{1,0.894,0.769}{#1}}%
\newcommand{\hlkwa}[1]{\textcolor[rgb]{0.941,0.902,0.549}{#1}}%
\newcommand{\hlkwb}[1]{\textcolor[rgb]{0.804,0.776,0.451}{#1}}%
\newcommand{\hlkwc}[1]{\textcolor[rgb]{0.78,0.941,0.545}{#1}}%
\newcommand{\hlkwd}[1]{\textcolor[rgb]{1,0.78,0.769}{#1}}%
\let\hlipl\hlkwb

\usepackage{framed}
\makeatletter
\newenvironment{kframe}{%
 \def\at@end@of@kframe{}%
 \ifinner\ifhmode%
  \def\at@end@of@kframe{\end{minipage}}%
  \begin{minipage}{\columnwidth}%
 \fi\fi%
 \def\FrameCommand##1{\hskip\@totalleftmargin \hskip-\fboxsep
 \colorbox{shadecolor}{##1}\hskip-\fboxsep
     % There is no \\@totalrightmargin, so:
     \hskip-\linewidth \hskip-\@totalleftmargin \hskip\columnwidth}%
 \MakeFramed {\advance\hsize-\width
   \@totalleftmargin\z@ \linewidth\hsize
   \@setminipage}}%
 {\par\unskip\endMakeFramed%
 \at@end@of@kframe}
\makeatother

\definecolor{shadecolor}{rgb}{.97, .97, .97}
\definecolor{messagecolor}{rgb}{0, 0, 0}
\definecolor{warningcolor}{rgb}{1, 0, 1}
\definecolor{errorcolor}{rgb}{1, 0, 0}
\newenvironment{knitrout}{}{} % an empty environment to be redefined in TeX

\usepackage{alltt}
\usepackage{../371g-slides}
\title{Simulation 3}
\subtitle{Lecture 24}
\author{STA 371G}
\IfFileExists{upquote.sty}{\usepackage{upquote}}{}
\begin{document}
  
  

  \frame{\maketitle}

  % Show outline at beginning of each section
  \AtBeginSection[]{
    \begin{frame}<beamer>
      \tableofcontents[currentsection]
    \end{frame}
  }

  %%%%%%% Slides start here %%%%%%%

  \begin{darkframes}
    \begin{frame}{Example 1: Simulating a betting game}
      \begin{itemize}[<+->]
        \item Suppose you are going to play blackjack, starting with \$100 in the ``bank.''
        \item In each hand, you can bet as much as you want.
        \item If you lose, you lose your bet.
        \item If you win, you win what you bet.
        \item Suppose there is a 40\% chance that you will win at any given hand.
        \item What will your bank look like after 20 hands of blackjack?
      \end{itemize}
    \end{frame}

    \begin{frame}[fragile]
      \begin{itemize}[<+->]
        \item Similar to \texttt{rnorm}, \texttt{runif} \underline{r}andomly samples from a \emph{\underline{unif}orm} distribution.
        \item Without any parameters, it samples from the range $[0,1]$.
\begin{knitrout}
\definecolor{shadecolor}{rgb}{0.137, 0.137, 0.137}\begin{kframe}
\begin{alltt}
\hlkwd{hist}\hlstd{(}\hlkwd{runif}\hlstd{(}\hlnum{100000}\hlstd{,} \hlkwc{min}\hlstd{=}\hlnum{4}\hlstd{,} \hlkwc{max}\hlstd{=}\hlnum{7}\hlstd{),} \hlkwc{col}\hlstd{=}\hlstr{"pink"}\hlstd{,} \hlkwc{main}\hlstd{=}\hlstr{""}\hlstd{)}
\end{alltt}
\end{kframe}
\input{/tmp/figures/unnamed-chunk-2-1.tex}

\end{knitrout}
      \end{itemize}
    \end{frame}

    \begin{frame}[fragile]{What is \texttt{runif} good for?}
      \begin{itemize}[<+->]
        \item \texttt{runif} is often useful for simulating an event that happens with a certain probability.
        \item For example, \texttt{runif(1)} randomly selects one number uniformly from $[0,1]$, so it will be less than 0.2 about 20\% of the time.
        \item To simulate an event that has a 20\% chance of happening:
\begin{knitrout}
\definecolor{shadecolor}{rgb}{0.137, 0.137, 0.137}\begin{kframe}
\begin{alltt}
\hlkwa{if} \hlstd{(}\hlkwd{runif}\hlstd{(}\hlnum{1}\hlstd{)} \hlopt{<} \hlnum{0.2}\hlstd{) \{}
  \hlcom{# The event happened.}
\hlstd{\}} \hlkwa{else} \hlstd{\{}
  \hlcom{# The event did not happen.}
\hlstd{\}}
\end{alltt}
\end{kframe}
\end{knitrout}
      \end{itemize}
    \end{frame}

    \begin{frame}[fragile]
      \fontsize{9}{9}\selectfont
\begin{knitrout}
\definecolor{shadecolor}{rgb}{0.137, 0.137, 0.137}\begin{kframe}
\begin{alltt}
\hlstd{results} \hlkwb{<-} \hlkwd{replicate}\hlstd{(}\hlnum{10000}\hlstd{, \{}
  \hlstd{bank} \hlkwb{<-} \hlnum{100}
  \hlstd{bet} \hlkwb{<-} \hlnum{1}
  \hlkwa{for} \hlstd{(hand} \hlkwa{in} \hlnum{1}\hlopt{:}\hlnum{20}\hlstd{) \{}
    \hlkwa{if} \hlstd{(}\hlkwd{runif}\hlstd{(}\hlnum{1}\hlstd{)} \hlopt{<} \hlnum{0.4}\hlstd{) \{}
      \hlstd{bank} \hlkwb{<-} \hlstd{bank} \hlopt{+} \hlstd{bet}
    \hlstd{\}} \hlkwa{else} \hlstd{\{}
      \hlstd{bank} \hlkwb{<-} \hlstd{bank} \hlopt{-} \hlstd{bet}
    \hlstd{\}}
  \hlstd{\}}
  \hlkwd{return}\hlstd{(bank)}
\hlstd{\})}
\hlkwd{hist}\hlstd{(results,} \hlkwc{col}\hlstd{=}\hlstr{"pink"}\hlstd{,} \hlkwc{main}\hlstd{=}\hlstr{""}\hlstd{)}
\end{alltt}
\end{kframe}
\input{/tmp/figures/unnamed-chunk-4-1.tex}

\end{knitrout}
    \end{frame}

    \begin{frame}
      \begin{itemize}
        \item Let's model another betting strategy. Suppose that if we lose, we double our bet in an effort to recover the previous loss.
        \item This way, if we lose, a win on the next hand would cancel out the losses.
        \item This is called a \emph{Martingale betting strategy}.
      \end{itemize}
    \end{frame}

    \begin{frame}[fragile]
      \fontsize{9}{9}\selectfont
\begin{knitrout}
\definecolor{shadecolor}{rgb}{0.137, 0.137, 0.137}\begin{kframe}
\begin{alltt}
\hlstd{results2} \hlkwb{<-} \hlkwd{replicate}\hlstd{(}\hlnum{10000}\hlstd{, \{}
  \hlstd{bank} \hlkwb{<-} \hlnum{100}
  \hlstd{bet} \hlkwb{<-} \hlnum{1}
  \hlkwa{for} \hlstd{(hand} \hlkwa{in} \hlnum{1}\hlopt{:}\hlnum{20}\hlstd{) \{}
    \hlstd{win} \hlkwb{<-} \hlstd{(}\hlkwd{runif}\hlstd{(}\hlnum{1}\hlstd{)} \hlopt{<} \hlnum{0.4}\hlstd{)}
    \hlkwa{if} \hlstd{(win) \{}
      \hlstd{bank} \hlkwb{<-} \hlstd{bank} \hlopt{+} \hlstd{bet}
    \hlstd{\}} \hlkwa{else} \hlstd{\{}
      \hlstd{bank} \hlkwb{<-} \hlstd{bank} \hlopt{-} \hlstd{bet}
      \hlstd{bet} \hlkwb{<-} \hlnum{2} \hlopt{*} \hlstd{bet}
    \hlstd{\}}
  \hlstd{\}}
  \hlkwd{return}\hlstd{(bank)}
\hlstd{\})}
\end{alltt}
\end{kframe}
\end{knitrout}
    \end{frame}

    \begin{frame}[fragile]
      
\begin{knitrout}
\definecolor{shadecolor}{rgb}{0.137, 0.137, 0.137}\begin{kframe}
\begin{alltt}
\hlkwd{hist}\hlstd{(results2,} \hlkwc{col}\hlstd{=}\hlstr{"pink"}\hlstd{,} \hlkwc{main}\hlstd{=}\hlstr{""}\hlstd{)}
\end{alltt}
\end{kframe}
\input{/tmp/figures/unnamed-chunk-7-1.tex}

\end{knitrout}
    \end{frame}

    \begin{frame}[fragile]{Examining percentiles of outcomes}
      \begin{itemize}[<+->]
        \item In addition to looking at the mean or SD of the simulated results, we can also examine percentiles.
        \item To look at the 5th, 50th (median), and 95th percentile result:
\begin{knitrout}
\definecolor{shadecolor}{rgb}{0.137, 0.137, 0.137}\begin{kframe}
\begin{alltt}
\hlkwd{quantile}\hlstd{(results2,} \hlkwc{probs}\hlstd{=}\hlkwd{c}\hlstd{(}\hlnum{.05}\hlstd{,} \hlnum{.5}\hlstd{,} \hlnum{.95}\hlstd{))}
\end{alltt}
\begin{verbatim}
    5%    50%    95% 
-31001    271  10409 
\end{verbatim}
\end{kframe}
\end{knitrout}
        \item In other words: under the Martingale strategy, 5\% of the time our bank will end up less than $-\$31001.1$, 50\% of the time we will end up with less than $\$271$, etc.
      \end{itemize}
    \end{frame}

    \begin{frame}{Comparing the strategies}
      \begin{center}
        \begin{tabular}{l|ll}
          Strategy & Original & Martingale \\
          \hline
          Expected total after 20 hands & 95.96 & -4026.97 \\
          5th percentile after 20 hands & 88 & -31001.1 \\
          95th percentile after 20 hands & 104 & 10409.1 \\
        \end{tabular}

        \pause\bigskip
        You could do better with the Martingale strategy, but you'll probably end up doing \emph{much} worse!
      \end{center}

    \end{frame}

    \begin{frame}{Example 2: Simulating pricing and demand}
      \begin{itemize}[<+->]
        \item Let's model the problem of oil drilling: you are planning to drill for oil in a newly-discovered field.
        \item Setting up the drilling equipment costs \$1M.
        \item There's a 45\% chance that you strike oil.
        \item If you strike oil, you will generate money---\pause but how much depends on the price of oil and how much demand there is.
      \end{itemize}
    \end{frame}

    \begin{frame}[fragile]
      \fontsm
      \begin{itemize}
        \item Suppose that the price of oil is normally distributed, with a mean of \$45/barrel and an SD of \$8.
        \item Suppose that the demand (the number of barrels we can sell) has a \emph{gamma distribution} with shape 3 and rate $1/40000$ (the gamma distribution can be used to model many processes, like waiting times and consumer demand, which must be $\geq 0$ and has a long right tail):
\begin{knitrout}
\definecolor{shadecolor}{rgb}{0.137, 0.137, 0.137}\begin{kframe}
\begin{alltt}
\hlkwd{hist}\hlstd{(}\hlkwd{rgamma}\hlstd{(}\hlnum{10000}\hlstd{,} \hlkwc{shape}\hlstd{=}\hlnum{3}\hlstd{,} \hlkwc{rate}\hlstd{=}\hlnum{1}\hlopt{/}\hlnum{40000}\hlstd{),}
  \hlkwc{col}\hlstd{=}\hlstr{"orange"}\hlstd{,} \hlkwc{main}\hlstd{=}\hlstr{""}\hlstd{)}
\end{alltt}
\end{kframe}
\input{/tmp/figures/unnamed-chunk-9-1.tex}

\end{knitrout}
      \end{itemize}

    \end{frame}

    \begin{frame}[fragile]
      \fontsm
\begin{knitrout}
\definecolor{shadecolor}{rgb}{0.137, 0.137, 0.137}\begin{kframe}
\begin{alltt}
\hlstd{results} \hlkwb{<-} \hlkwd{replicate}\hlstd{(}\hlnum{10000}\hlstd{, \{}
  \hlkwa{if} \hlstd{(}\hlkwd{runif}\hlstd{(}\hlnum{1}\hlstd{)} \hlopt{<} \hlnum{0.45}\hlstd{) \{}
    \hlstd{price} \hlkwb{<-} \hlkwd{rnorm}\hlstd{(}\hlnum{1}\hlstd{,} \hlkwc{mean}\hlstd{=}\hlnum{45}\hlstd{,} \hlkwc{sd}\hlstd{=}\hlnum{8}\hlstd{)}
    \hlstd{demand} \hlkwb{<-} \hlkwd{rgamma}\hlstd{(}\hlnum{1}\hlstd{,} \hlkwc{shape}\hlstd{=}\hlnum{3}\hlstd{,} \hlkwc{rate}\hlstd{=}\hlnum{1}\hlopt{/}\hlnum{40000}\hlstd{)}
    \hlstd{revenue} \hlkwb{<-} \hlstd{price} \hlopt{*} \hlstd{demand} \hlopt{-} \hlnum{1000000}
  \hlstd{\}} \hlkwa{else} \hlstd{\{}
    \hlstd{revenue} \hlkwb{<-} \hlopt{-}\hlnum{1000000}
  \hlstd{\}}
  \hlkwd{return}\hlstd{(revenue)}
\hlstd{\})}
\hlkwd{hist}\hlstd{(results,} \hlkwc{col}\hlstd{=}\hlstr{"lightblue"}\hlstd{,} \hlkwc{main}\hlstd{=}\hlstr{""}\hlstd{)}
\end{alltt}
\end{kframe}
\input{/tmp/figures/unnamed-chunk-10-1.tex}

\end{knitrout}
    \end{frame}

    \begin{frame}
      \begin{itemize}[<+->]
        \item We have been assuming that the amount of oil produced by the field is always sufficient to meet the demand.
        \item It's more realistic to not assume this will necessarily be the case.
        \item Suppose that production also has a gamma distribution with shape 3 and rate $1/40000$.
        \item How much do we make if we decide to drill for oil?
      \end{itemize}

    \end{frame}

    \begin{frame}[fragile]
      \fontvsm
\begin{knitrout}
\definecolor{shadecolor}{rgb}{0.137, 0.137, 0.137}\begin{kframe}
\begin{alltt}
\hlstd{results} \hlkwb{<-} \hlkwd{replicate}\hlstd{(}\hlnum{10000}\hlstd{, \{}
  \hlkwa{if} \hlstd{(}\hlkwd{runif}\hlstd{(}\hlnum{1}\hlstd{)} \hlopt{<} \hlnum{0.45}\hlstd{) \{}
    \hlstd{price} \hlkwb{<-} \hlkwd{rnorm}\hlstd{(}\hlnum{1}\hlstd{,} \hlkwc{mean}\hlstd{=}\hlnum{45}\hlstd{,} \hlkwc{sd}\hlstd{=}\hlnum{8}\hlstd{)}
    \hlstd{demand} \hlkwb{<-} \hlkwd{rgamma}\hlstd{(}\hlnum{1}\hlstd{,} \hlkwc{shape}\hlstd{=}\hlnum{3}\hlstd{,} \hlkwc{rate}\hlstd{=}\hlnum{1}\hlopt{/}\hlnum{40000}\hlstd{)}
    \hlstd{production} \hlkwb{<-} \hlkwd{rgamma}\hlstd{(}\hlnum{1}\hlstd{,} \hlkwc{shape}\hlstd{=}\hlnum{3}\hlstd{,} \hlkwc{rate}\hlstd{=}\hlnum{1}\hlopt{/}\hlnum{40000}\hlstd{)}
    \hlstd{barrels.sold} \hlkwb{<-} \hlkwd{min}\hlstd{(production, demand)}
    \hlstd{revenue} \hlkwb{<-} \hlstd{price} \hlopt{*} \hlstd{barrels.sold} \hlopt{-} \hlnum{1000000}
  \hlstd{\}} \hlkwa{else} \hlstd{\{}
    \hlstd{revenue} \hlkwb{<-} \hlopt{-}\hlnum{1000000}
  \hlstd{\}}
  \hlkwd{return}\hlstd{(revenue)}
\hlstd{\})}
\hlkwd{hist}\hlstd{(results,} \hlkwc{col}\hlstd{=}\hlstr{"lightblue"}\hlstd{,} \hlkwc{main}\hlstd{=}\hlstr{""}\hlstd{)}
\end{alltt}
\end{kframe}
\input{/tmp/figures/unnamed-chunk-11-1.tex}

\end{knitrout}
    \end{frame}
  \end{darkframes}
\end{document}

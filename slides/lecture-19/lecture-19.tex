\documentclass{beamer}\usepackage[]{graphicx}\usepackage[]{color}
%% maxwidth is the original width if it is less than linewidth
%% otherwise use linewidth (to make sure the graphics do not exceed the margin)
\makeatletter
\def\maxwidth{ %
  \ifdim\Gin@nat@width>\linewidth
    \linewidth
  \else
    \Gin@nat@width
  \fi
}
\makeatother

\definecolor{fgcolor}{rgb}{1, 0.894, 0.769}
\newcommand{\hlnum}[1]{\textcolor[rgb]{0.824,0.412,0.118}{#1}}%
\newcommand{\hlstr}[1]{\textcolor[rgb]{1,0.894,0.71}{#1}}%
\newcommand{\hlcom}[1]{\textcolor[rgb]{0.824,0.706,0.549}{#1}}%
\newcommand{\hlopt}[1]{\textcolor[rgb]{1,0.894,0.769}{#1}}%
\newcommand{\hlstd}[1]{\textcolor[rgb]{1,0.894,0.769}{#1}}%
\newcommand{\hlkwa}[1]{\textcolor[rgb]{0.941,0.902,0.549}{#1}}%
\newcommand{\hlkwb}[1]{\textcolor[rgb]{0.804,0.776,0.451}{#1}}%
\newcommand{\hlkwc}[1]{\textcolor[rgb]{0.78,0.941,0.545}{#1}}%
\newcommand{\hlkwd}[1]{\textcolor[rgb]{1,0.78,0.769}{#1}}%
\let\hlipl\hlkwb

\usepackage{framed}
\makeatletter
\newenvironment{kframe}{%
 \def\at@end@of@kframe{}%
 \ifinner\ifhmode%
  \def\at@end@of@kframe{\end{minipage}}%
  \begin{minipage}{\columnwidth}%
 \fi\fi%
 \def\FrameCommand##1{\hskip\@totalleftmargin \hskip-\fboxsep
 \colorbox{shadecolor}{##1}\hskip-\fboxsep
     % There is no \\@totalrightmargin, so:
     \hskip-\linewidth \hskip-\@totalleftmargin \hskip\columnwidth}%
 \MakeFramed {\advance\hsize-\width
   \@totalleftmargin\z@ \linewidth\hsize
   \@setminipage}}%
 {\par\unskip\endMakeFramed%
 \at@end@of@kframe}
\makeatother

\definecolor{shadecolor}{rgb}{.97, .97, .97}
\definecolor{messagecolor}{rgb}{0, 0, 0}
\definecolor{warningcolor}{rgb}{1, 0, 1}
\definecolor{errorcolor}{rgb}{1, 0, 0}
\newenvironment{knitrout}{}{} % an empty environment to be redefined in TeX

\usepackage{alltt}
\usepackage{preview}
\usepackage{../371g-slides}
\title{Logistic regression lab}
\subtitle{Lecture 19}
\author{STA 371G}
\IfFileExists{upquote.sty}{\usepackage{upquote}}{}
\begin{document}
  
  

  \frame{\maketitle}

  % Show outline at beginning of each section
  \AtBeginSection[]{
    \begin{frame}<beamer>
      \tableofcontents[currentsection]
    \end{frame}
  }

  %%%%%%% Slides start here %%%%%%%

  \begin{darkframes}
    \begin{frame}
      \begin{center}
        No Learning Catalytics questions yet....

        \bigskip

        \includegraphics[width=1.8in,keepaspectratio]{null_hypothesis}
      \end{center}
    \end{frame}

    \begin{frame}{Case study: An application of logistic regression}
      \begin{itemize}[<+->]
        \item Imagine you are running a web site and you are considering whether to present an offer for a magazine subscription to users.
        \item Presenting the offer when the customer is not interested will annoy them; not presenting the offer means you forgo a possible sale.
        \item How do you decide which customers to present the offer to?
      \end{itemize}
    \end{frame}

    \begin{frame}{Building the data set}
      \begin{itemize}[<+->]
        \item The company recently concluded an email campaign where they offered the \emph{Creativity for Kids} magazine to the email list.
        \item An email clickthrough tracked whether each customer subscribed to the magazine or not.
        \item The company matched the data collected when the customer made a previous purchase with third-party data (which can be purchased from data sources such as the credit scoring agencies).
      \end{itemize}
    \end{frame}

    \begin{frame}{The activity}
      Your goal is to build a logistic regression model to help the company make a prediction about whether to show the offer to a customer on the web site, based on characteristic of that customer:
      \begin{itemize}
        \item Demographics (income, gender, marital status, etc.)
        \item Previous history with the company (previously purchased a parenting magazine; previously purchased a children's magazine)
      \end{itemize}
    \end{frame}
  \end{darkframes}
\end{document}

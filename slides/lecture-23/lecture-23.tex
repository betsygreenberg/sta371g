\documentclass{beamer}\usepackage[]{graphicx}\usepackage[]{color}
%% maxwidth is the original width if it is less than linewidth
%% otherwise use linewidth (to make sure the graphics do not exceed the margin)
\makeatletter
\def\maxwidth{ %
  \ifdim\Gin@nat@width>\linewidth
    \linewidth
  \else
    \Gin@nat@width
  \fi
}
\makeatother

\definecolor{fgcolor}{rgb}{1, 0.894, 0.769}
\newcommand{\hlnum}[1]{\textcolor[rgb]{0.824,0.412,0.118}{#1}}%
\newcommand{\hlstr}[1]{\textcolor[rgb]{1,0.894,0.71}{#1}}%
\newcommand{\hlcom}[1]{\textcolor[rgb]{0.824,0.706,0.549}{#1}}%
\newcommand{\hlopt}[1]{\textcolor[rgb]{1,0.894,0.769}{#1}}%
\newcommand{\hlstd}[1]{\textcolor[rgb]{1,0.894,0.769}{#1}}%
\newcommand{\hlkwa}[1]{\textcolor[rgb]{0.941,0.902,0.549}{#1}}%
\newcommand{\hlkwb}[1]{\textcolor[rgb]{0.804,0.776,0.451}{#1}}%
\newcommand{\hlkwc}[1]{\textcolor[rgb]{0.78,0.941,0.545}{#1}}%
\newcommand{\hlkwd}[1]{\textcolor[rgb]{1,0.78,0.769}{#1}}%
\let\hlipl\hlkwb

\usepackage{framed}
\makeatletter
\newenvironment{kframe}{%
 \def\at@end@of@kframe{}%
 \ifinner\ifhmode%
  \def\at@end@of@kframe{\end{minipage}}%
  \begin{minipage}{\columnwidth}%
 \fi\fi%
 \def\FrameCommand##1{\hskip\@totalleftmargin \hskip-\fboxsep
 \colorbox{shadecolor}{##1}\hskip-\fboxsep
     % There is no \\@totalrightmargin, so:
     \hskip-\linewidth \hskip-\@totalleftmargin \hskip\columnwidth}%
 \MakeFramed {\advance\hsize-\width
   \@totalleftmargin\z@ \linewidth\hsize
   \@setminipage}}%
 {\par\unskip\endMakeFramed%
 \at@end@of@kframe}
\makeatother

\definecolor{shadecolor}{rgb}{.97, .97, .97}
\definecolor{messagecolor}{rgb}{0, 0, 0}
\definecolor{warningcolor}{rgb}{1, 0, 1}
\definecolor{errorcolor}{rgb}{1, 0, 0}
\newenvironment{knitrout}{}{} % an empty environment to be redefined in TeX

\usepackage{alltt}
\usepackage{../371g-slides}
\title{Simulation 2}
\subtitle{Lecture 23}
\author{STA 371G}
\IfFileExists{upquote.sty}{\usepackage{upquote}}{}
\begin{document}
  
  

  \frame{\maketitle}

  % Show outline at beginning of each section
  \AtBeginSection[]{
    \begin{frame}<beamer>
      \tableofcontents[currentsection]
    \end{frame}
  }

  %%%%%%% Slides start here %%%%%%%

  \begin{darkframes}
    \begin{frame}{Announcements}
      \begin{itemize}
        \item HW 8 (last one!) is posted and is due Tuesday, May 7.
        \item Peer evaluations for Part 3 of the project are due this Friday.
        \item Part 4 of the project is due this Friday.
      \end{itemize}
    \end{frame}

    \begin{frame}{Will you have enough money for retirement?}
      \begin{itemize}[<+->]
        \item Most people (try to) save for retirement throughout their career.
        \item But how do you know that the money you have saved will last you from retirement age until death?
        \item Particularly with expected lifespans growing, how do you know that you won't outlive your money?
      \end{itemize}
    \end{frame}

    \begin{frame}{Will you have enough money for retirement?}
      \begin{itemize}[<+->]
        \item Let's start by building a simple model of a portfolio.
        \item Each year, at the beginning of the year, I put \$10,000 into a retirement account.
        \item The entire account will be invested in a single asset that has normally distributed annual returns, with mean 12\% and SD 18\%.
        \item What will the portfolio look like in 30 years?
      \end{itemize}
    \end{frame}

    \begin{frame}[fragile]
      Let's start by writing R code for the first year:
\begin{knitrout}
\definecolor{shadecolor}{rgb}{0.137, 0.137, 0.137}\begin{kframe}
\begin{alltt}
\hlcom{# Start with nothing}
\hlstd{account.value} \hlkwb{<-} \hlnum{0}
\hlcom{# Add a $10,000 investment}
\hlstd{account.value} \hlkwb{<-} \hlstd{account.value} \hlopt{+} \hlnum{10000}
\hlcom{# Simulate this year's return}
\hlstd{this.years.return} \hlkwb{<-} \hlkwd{rnorm}\hlstd{(}\hlnum{1}\hlstd{,} \hlkwc{mean}\hlstd{=}\hlnum{.12}\hlstd{,} \hlkwc{sd}\hlstd{=}\hlnum{.18}\hlstd{)}
\hlcom{# Apply the return}
\hlstd{account.value} \hlkwb{<-} \hlstd{account.value} \hlopt{*} \hlstd{(}\hlnum{1} \hlopt{+} \hlstd{this.years.return)}
\hlcom{# Examine the account value}
\hlstd{account.value}
\end{alltt}
\begin{verbatim}
[1] 10072.38
\end{verbatim}
\end{kframe}
\end{knitrout}
    \end{frame}

    \begin{frame}[fragile]{Repeating a process within a run}
      \fontsm
      \begin{itemize}[<+->]
        \item Within a single run of our simulation, we need to simulate 30 years of saving.
        \item To do this, we can use a \texttt{for} loop, which lets us run a block of code repeatedly; for example, we can print out the first 5 square numbers:
\begin{knitrout}
\definecolor{shadecolor}{rgb}{0.137, 0.137, 0.137}\begin{kframe}
\begin{alltt}
\hlkwa{for} \hlstd{(n} \hlkwa{in} \hlnum{1}\hlopt{:}\hlnum{5}\hlstd{) \{}
  \hlkwd{print}\hlstd{(n}\hlopt{^}\hlnum{2}\hlstd{)}
\hlstd{\}}
\end{alltt}
\begin{verbatim}
[1] 1
[1] 4
[1] 9
[1] 16
[1] 25
\end{verbatim}
\end{kframe}
\end{knitrout}
        The value of \verb|n| is set to a different value on each iteration of the code block \verb|{ ... }|.
      \end{itemize}
    \end{frame}

    \begin{frame}[fragile]
      Returning to our portfolio simulation, let's repeat the process for 30 years:
      \fontsm
\begin{knitrout}
\definecolor{shadecolor}{rgb}{0.137, 0.137, 0.137}\begin{kframe}
\begin{alltt}
\hlstd{account.value} \hlkwb{<-} \hlnum{0}
\hlkwa{for} \hlstd{(year} \hlkwa{in} \hlnum{1}\hlopt{:}\hlnum{30}\hlstd{) \{}
  \hlstd{account.value} \hlkwb{<-} \hlstd{account.value} \hlopt{+} \hlnum{10000}
  \hlstd{this.years.return} \hlkwb{<-} \hlkwd{rnorm}\hlstd{(}\hlnum{1}\hlstd{,} \hlkwc{mean}\hlstd{=}\hlnum{.12}\hlstd{,} \hlkwc{sd}\hlstd{=}\hlnum{.18}\hlstd{)}
  \hlstd{account.value} \hlkwb{<-} \hlstd{account.value} \hlopt{*} \hlstd{(}\hlnum{1} \hlopt{+} \hlstd{this.years.return)}
\hlstd{\}}
\hlstd{account.value}
\end{alltt}
\begin{verbatim}
[1] 3330261
\end{verbatim}
\end{kframe}
\end{knitrout}
    \end{frame}

    \begin{frame}[fragile]
      \fontsize{9}{9}\selectfont
\begin{knitrout}
\definecolor{shadecolor}{rgb}{0.137, 0.137, 0.137}\begin{kframe}
\begin{alltt}
\hlstd{portfolio.values} \hlkwb{<-} \hlkwd{replicate}\hlstd{(}\hlnum{10000}\hlstd{, \{}
  \hlstd{account.value} \hlkwb{<-} \hlnum{0}
  \hlkwa{for} \hlstd{(year} \hlkwa{in} \hlnum{1}\hlopt{:}\hlnum{30}\hlstd{) \{}
    \hlstd{account.value} \hlkwb{<-} \hlstd{account.value} \hlopt{+} \hlnum{10000}
    \hlstd{this.years.return} \hlkwb{<-} \hlkwd{rnorm}\hlstd{(}\hlnum{1}\hlstd{,} \hlkwc{mean}\hlstd{=}\hlnum{.12}\hlstd{,} \hlkwc{sd}\hlstd{=}\hlnum{.18}\hlstd{)}
    \hlstd{account.value} \hlkwb{<-} \hlstd{account.value} \hlopt{*} \hlstd{(}\hlnum{1} \hlopt{+} \hlstd{this.years.return)}
  \hlstd{\}}
  \hlkwd{return}\hlstd{(account.value)}
\hlstd{\})}
\hlkwd{hist}\hlstd{(portfolio.values,} \hlkwc{col}\hlstd{=}\hlstr{"lightblue"}\hlstd{)}
\end{alltt}
\end{kframe}
\input{/tmp/figures/unnamed-chunk-5-1.tex}

\end{knitrout}
      \lc
    \end{frame}

    \begin{frame}[fragile]
      This looks pretty good at first, but there is a wide range of outcomes---let's look at the percentiles:
\begin{knitrout}
\definecolor{shadecolor}{rgb}{0.137, 0.137, 0.137}\begin{kframe}
\begin{alltt}
\hlkwd{quantile}\hlstd{(portfolio.values,}
  \hlkwc{probs}\hlstd{=}\hlkwd{c}\hlstd{(}\hlnum{0.05}\hlstd{,} \hlnum{0.25}\hlstd{,} \hlnum{0.5}\hlstd{,} \hlnum{0.75}\hlstd{,} \hlnum{0.95}\hlstd{))}
\end{alltt}
\begin{verbatim}
       5%       25%       50%       75%       95% 
 674417.4 1300215.9 2087403.8 3345929.9 6745048.2 
\end{verbatim}
\end{kframe}
\end{knitrout}
      That's a ~10x spread between the 95th percentile outcome and the 5th percentile outcome!

      \pause
      \bigskip
      The expected (average) value of my portfolio is:
\begin{knitrout}
\definecolor{shadecolor}{rgb}{0.137, 0.137, 0.137}\begin{kframe}
\begin{alltt}
\hlkwd{mean}\hlstd{(portfolio.values)}
\end{alltt}
\begin{verbatim}
[1] 2683627
\end{verbatim}
\end{kframe}
\end{knitrout}
    \end{frame}

    \begin{frame}[fragile]{What if I only contribute \$5,000 per year?}
      \fontsize{9}{9}\selectfont
\begin{knitrout}
\definecolor{shadecolor}{rgb}{0.137, 0.137, 0.137}\begin{kframe}
\begin{alltt}
\hlstd{portfolio.values} \hlkwb{<-} \hlkwd{replicate}\hlstd{(}\hlnum{10000}\hlstd{, \{}
  \hlstd{account.value} \hlkwb{<-} \hlnum{0}
  \hlkwa{for} \hlstd{(year} \hlkwa{in} \hlnum{1}\hlopt{:}\hlnum{30}\hlstd{) \{}
    \hlstd{account.value} \hlkwb{<-} \hlstd{account.value} \hlopt{+} \hlnum{5000}
    \hlstd{this.years.return} \hlkwb{<-} \hlkwd{rnorm}\hlstd{(}\hlnum{1}\hlstd{,} \hlkwc{mean}\hlstd{=}\hlnum{.12}\hlstd{,} \hlkwc{sd}\hlstd{=}\hlnum{.18}\hlstd{)}
    \hlstd{account.value} \hlkwb{<-} \hlstd{account.value} \hlopt{*} \hlstd{(}\hlnum{1} \hlopt{+} \hlstd{this.years.return)}
  \hlstd{\}}
  \hlkwd{return}\hlstd{(account.value)}
\hlstd{\})}
\end{alltt}
\end{kframe}
\end{knitrout}
    \end{frame}

    \begin{frame}[fragile]
      Let's look at the percentiles, under this new scenario where I only contribute \$5,000 per year:
\begin{knitrout}
\definecolor{shadecolor}{rgb}{0.137, 0.137, 0.137}\begin{kframe}
\begin{alltt}
\hlkwd{quantile}\hlstd{(portfolio.values,}
  \hlkwc{probs}\hlstd{=}\hlkwd{c}\hlstd{(}\hlnum{0.05}\hlstd{,} \hlnum{0.25}\hlstd{,} \hlnum{0.5}\hlstd{,} \hlnum{0.75}\hlstd{,} \hlnum{0.95}\hlstd{))}
\end{alltt}
\begin{verbatim}
       5%       25%       50%       75%       95% 
 331238.9  661227.0 1047610.8 1697519.4 3368771.5 
\end{verbatim}
\end{kframe}
\end{knitrout}

      \bigskip
      The expected (average) value of my portfolio is:
\begin{knitrout}
\definecolor{shadecolor}{rgb}{0.137, 0.137, 0.137}\begin{kframe}
\begin{alltt}
\hlkwd{mean}\hlstd{(portfolio.values)}
\end{alltt}
\begin{verbatim}
[1] 1344782
\end{verbatim}
\end{kframe}
\end{knitrout}

      \pause\bigskip

      Not contributing that extra \$5,000 per year is expected to cost me ~\$1.3M in the value of my retirement account!
    \end{frame}

    \begin{frame}[fragile]{How likely is it we'll retire a millionaire?}
      \fontsize{9}{9}\selectfont
\begin{knitrout}
\definecolor{shadecolor}{rgb}{0.137, 0.137, 0.137}\begin{kframe}
\begin{alltt}
\hlstd{results} \hlkwb{<-} \hlkwd{replicate}\hlstd{(}\hlnum{10000}\hlstd{, \{}
  \hlstd{account.value} \hlkwb{<-} \hlnum{0}
  \hlkwa{for} \hlstd{(year} \hlkwa{in} \hlnum{1}\hlopt{:}\hlnum{30}\hlstd{) \{}
    \hlstd{account.value} \hlkwb{<-} \hlstd{account.value} \hlopt{+} \hlnum{10000}
    \hlstd{this.years.return} \hlkwb{<-} \hlkwd{rnorm}\hlstd{(}\hlnum{1}\hlstd{,} \hlkwc{mean}\hlstd{=}\hlnum{.12}\hlstd{,} \hlkwc{sd}\hlstd{=}\hlnum{.18}\hlstd{)}
    \hlstd{account.value} \hlkwb{<-} \hlstd{account.value} \hlopt{*} \hlstd{(}\hlnum{1} \hlopt{+} \hlstd{this.years.return)}
  \hlstd{\}}
  \hlkwd{return}\hlstd{(account.value} \hlopt{>=} \hlnum{1000000}\hlstd{)}
\hlstd{\})}
\hlkwd{sum}\hlstd{(results)} \hlopt{/} \hlnum{10000}
\end{alltt}
\begin{verbatim}
[1] 0.8591
\end{verbatim}
\end{kframe}
\end{knitrout}
    \end{frame}

    \begin{frame}{How likely is it that I'll outlive my money?}
      \begin{itemize}[<+->]
        \item Suppose that after I turn 67 (i.e., year 30), I start withdrawing \$100,000 each year to live on, and I stop making annual contributions.
        \item Let's consider a random 37-year old male, nonsmoker, moderately active, always wear seat belt, statistics fanatic, etc.---his life expectancy is \pause 88.
        \item So let's simulate this process, and see how often my money outlives me.
      \end{itemize}
    \end{frame}

    \begin{frame}[fragile]
      \fontvsm
\begin{knitrout}
\definecolor{shadecolor}{rgb}{0.137, 0.137, 0.137}\begin{kframe}
\begin{alltt}
\hlstd{results} \hlkwb{<-} \hlkwd{replicate}\hlstd{(}\hlnum{10000}\hlstd{, \{}
  \hlstd{account.value} \hlkwb{<-} \hlnum{0}
  \hlkwa{for} \hlstd{(age} \hlkwa{in} \hlnum{37}\hlopt{:}\hlnum{66}\hlstd{) \{}
    \hlstd{account.value} \hlkwb{<-} \hlstd{account.value} \hlopt{+} \hlnum{10000}
    \hlstd{this.years.return} \hlkwb{<-} \hlkwd{rnorm}\hlstd{(}\hlnum{1}\hlstd{,} \hlkwc{mean}\hlstd{=}\hlnum{.12}\hlstd{,} \hlkwc{sd}\hlstd{=}\hlnum{.18}\hlstd{)}
    \hlstd{account.value} \hlkwb{<-} \hlstd{account.value} \hlopt{*} \hlstd{(}\hlnum{1} \hlopt{+} \hlstd{this.years.return)}
  \hlstd{\}}
  \hlkwa{for} \hlstd{(age} \hlkwa{in} \hlnum{67}\hlopt{:}\hlnum{88}\hlstd{) \{}
    \hlstd{account.value} \hlkwb{<-} \hlstd{account.value} \hlopt{-} \hlnum{100000}
    \hlkwa{if} \hlstd{(account.value} \hlopt{<} \hlnum{0}\hlstd{) \{}
      \hlstd{account.value} \hlkwb{<-} \hlnum{0}
    \hlstd{\}} \hlkwa{else} \hlstd{\{}
      \hlstd{this.years.return} \hlkwb{<-} \hlkwd{rnorm}\hlstd{(}\hlnum{1}\hlstd{,} \hlkwc{mean}\hlstd{=}\hlnum{.12}\hlstd{,} \hlkwc{sd}\hlstd{=}\hlnum{.18}\hlstd{)}
      \hlstd{account.value} \hlkwb{<-} \hlstd{account.value} \hlopt{*} \hlstd{(}\hlnum{1} \hlopt{+} \hlstd{this.years.return)}
    \hlstd{\}}
  \hlstd{\}}
  \hlkwd{return}\hlstd{(account.value} \hlopt{>} \hlnum{0}\hlstd{)}
\hlstd{\})}
\hlkwd{sum}\hlstd{(results)} \hlopt{/} \hlnum{10000}
\end{alltt}
\begin{verbatim}
[1] 0.843
\end{verbatim}
\end{kframe}
\end{knitrout}
      \pause
      There is a 84.3\% chance that I don't run out of money during retirement.
    \end{frame}
  \end{darkframes}
\end{document}

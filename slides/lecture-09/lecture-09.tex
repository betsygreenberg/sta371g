\documentclass{beamer}\usepackage[]{graphicx}\usepackage[]{color}
%% maxwidth is the original width if it is less than linewidth
%% otherwise use linewidth (to make sure the graphics do not exceed the margin)
\makeatletter
\def\maxwidth{ %
  \ifdim\Gin@nat@width>\linewidth
    \linewidth
  \else
    \Gin@nat@width
  \fi
}
\makeatother

\definecolor{fgcolor}{rgb}{1, 0.894, 0.769}
\newcommand{\hlnum}[1]{\textcolor[rgb]{0.824,0.412,0.118}{#1}}%
\newcommand{\hlstr}[1]{\textcolor[rgb]{1,0.894,0.71}{#1}}%
\newcommand{\hlcom}[1]{\textcolor[rgb]{0.824,0.706,0.549}{#1}}%
\newcommand{\hlopt}[1]{\textcolor[rgb]{1,0.894,0.769}{#1}}%
\newcommand{\hlstd}[1]{\textcolor[rgb]{1,0.894,0.769}{#1}}%
\newcommand{\hlkwa}[1]{\textcolor[rgb]{0.941,0.902,0.549}{#1}}%
\newcommand{\hlkwb}[1]{\textcolor[rgb]{0.804,0.776,0.451}{#1}}%
\newcommand{\hlkwc}[1]{\textcolor[rgb]{0.78,0.941,0.545}{#1}}%
\newcommand{\hlkwd}[1]{\textcolor[rgb]{1,0.78,0.769}{#1}}%
\let\hlipl\hlkwb

\usepackage{framed}
\makeatletter
\newenvironment{kframe}{%
 \def\at@end@of@kframe{}%
 \ifinner\ifhmode%
  \def\at@end@of@kframe{\end{minipage}}%
  \begin{minipage}{\columnwidth}%
 \fi\fi%
 \def\FrameCommand##1{\hskip\@totalleftmargin \hskip-\fboxsep
 \colorbox{shadecolor}{##1}\hskip-\fboxsep
     % There is no \\@totalrightmargin, so:
     \hskip-\linewidth \hskip-\@totalleftmargin \hskip\columnwidth}%
 \MakeFramed {\advance\hsize-\width
   \@totalleftmargin\z@ \linewidth\hsize
   \@setminipage}}%
 {\par\unskip\endMakeFramed%
 \at@end@of@kframe}
\makeatother

\definecolor{shadecolor}{rgb}{.97, .97, .97}
\definecolor{messagecolor}{rgb}{0, 0, 0}
\definecolor{warningcolor}{rgb}{1, 0, 1}
\definecolor{errorcolor}{rgb}{1, 0, 0}
\newenvironment{knitrout}{}{} % an empty environment to be redefined in TeX

\usepackage{alltt}
\usepackage{../371g-slides}
\usepackage{dcolumn}
\title{Model building: selecting a model 1}
\subtitle{Lecture 9}
\author{STA 371G}
\IfFileExists{upquote.sty}{\usepackage{upquote}}{}
\begin{document}
  
  

  \frame{\maketitle}

  % Show outline at beginning of each section
  \AtBeginSection[]{
    \begin{frame}<beamer>
      \tableofcontents[currentsection]
    \end{frame}
  }

  %%%%%%% Slides start here %%%%%%%

  \begin{darkframes}
    \begin{frame}{Midterm 1: Wednesday, February 27}
      \begin{itemize}[<+->]
        \item The midterm will be in this room; you'll have 75 minutes.
        \item Bring your laptop (or borrow one from the Media Center) to take the test on.
        \item You can use RStudio, the R documentation, and the R help pages as reference during the test.
        \item You can also bring 1 page of your own notes to use during the test.
        \item Do not spend a lot of time re-reading the book/notes; the best way to study is \alert{practicing by working new problems}:
          \begin{itemize}
            \item Extra credit problem set on MyStatLab
            \item Practice exam on Canvas
            \item More practice problems in MyStatLab: select Study Plan > All Chapters on the left menu
          \end{itemize}
      \end{itemize}
    \end{frame}

    \begin{frame}
      \fullpagepicture{doc-shortage}
    \end{frame}

    \begin{frame}{What might explain why some counties have a doctor shortage?}
      \begin{columns}[onlytextwidth]
        \column{.5\textwidth}
          \begin{itemize}
            \item Small counties
            \item Poverty
            \item Health insurance
          \end{itemize}
        \column{.5\textwidth}
          \begin{itemize}
            \item Unemployment
            \item Large rural areas
            \item Something else?
          \end{itemize}
      \end{columns}
    \end{frame}

    \begin{frame}{This is a different use of regression}
      \begin{itemize}[<+->]
        \item The purpose of this regression is not to make predictions---there are only 254 counties in Texas and we have data on all of them.
        \item Instead, we are using regression here to \alert{understand the underlying factors} that explain doctor shortages.
      \end{itemize}
    \end{frame}

    \begin{frame}[fragile]{Population as a predictor of number of physicians}
\begin{knitrout}
\definecolor{shadecolor}{rgb}{0.137, 0.137, 0.137}\begin{kframe}
\begin{alltt}
\hlstd{> }\hlstd{popmodel} \hlkwb{<-} \hlkwd{lm}\hlstd{(Physicians} \hlopt{~} \hlstd{Population,} \hlkwc{data}\hlstd{=counties)}
\hlstd{> }\hlkwd{plot}\hlstd{(counties}\hlopt{$}\hlstd{Population, counties}\hlopt{$}\hlstd{Physicians)}
\hlstd{> }\hlkwd{abline}\hlstd{(popmodel)}
\end{alltt}
\end{kframe}
\input{/tmp/figures/unnamed-chunk-2-1.tex}

\end{knitrout}
    \end{frame}

    \begin{frame}[fragile]{Transform and Subset the data}
      Let's define a new variable for physicians per 10,000 people---this is important as absolute numbers aren't really what we care about (large counties have lots of doctors, which isn't a helpful fact!):
\begin{knitrout}
\definecolor{shadecolor}{rgb}{0.137, 0.137, 0.137}\begin{kframe}
\begin{alltt}
\hlstd{> }\hlstd{counties}\hlopt{$}\hlstd{PhysiciansPer10000} \hlkwb{<-}
\hlstd{+ }  \hlstd{counties}\hlopt{$}\hlstd{Physicians} \hlopt{/} \hlstd{counties}\hlopt{$}\hlstd{Population} \hlopt{*} \hlnum{10000}
\end{alltt}
\end{kframe}
\end{knitrout}

      \pause

      Then let's remove the very small counties as we can't reliably measure physician density in small counties:
\begin{knitrout}
\definecolor{shadecolor}{rgb}{0.137, 0.137, 0.137}\begin{kframe}
\begin{alltt}
\hlstd{> }\hlstd{my.counties} \hlkwb{<-} \hlkwd{subset}\hlstd{(counties, Population} \hlopt{>} \hlnum{10000}\hlstd{)}
\end{alltt}
\end{kframe}
\end{knitrout}
    \end{frame}

    \begin{frame}[fragile]{Potential predictor variables}
      \begin{itemize}
        \item \textbf{LandArea}:       Area in square miles
        \item \textbf{PctRural}:       Percentage rural land
        \item \textbf{MedianIncome}:   Median household income
        \item \textbf{Population}:     Population
        \item \textbf{PctUnder18}:     Percent children
        \item \textbf{PctOver65}:      Percent seniors
        \item \textbf{PctPoverty}:     Percent below the poverty line
        \item \textbf{PctUninsured}:   Percent without health insurance
        \item \textbf{PctSomeCollege}: Percent with some higher education
        \item \textbf{PctUnemployed}:  Percent unemployed
      \end{itemize}
    \end{frame}

    \begin{frame}{Parsimony}
      \begin{itemize}[<+->]
        \item We want a model that has a high $R^2$ and a low $s_e$, because then the predictors are doing a good job of explaining $Y$---and our predictions will be more accurate.
        \item We also want a model that is simple, so it's easy to explain to a non-expert.
        \item The ideal model is \alert{parsimonious}: a good trade-off between simplicity (as few variables as possible) and a high $R^2$.
      \end{itemize}
    \end{frame}

    \begin{frame}
      \begin{center}
        There is no purely mechanical procedure that will tell you what the most parsimonious model is; you need to use your judgement.

        \bigskip\pause

        But with $k$ variables there are $2^k-1$ possible models; for example, there are $k=10$ possible predictor variables in the data set, so there are 1,023 possible combinations of predictors you could use!
      \end{center}
    \end{frame}

    \begin{frame}{General strategy}
      \begin{enumerate}[<+->]
        \item Use one or more procedures to generate candidate models: possible models that are worth considering.
        \item Select the candidate model with a reasonable tradeoff simplicity and predictive power (high $R^2$).
        \item Check assumptions and model diagnostics (more on this to come); apply transformations and other fixes if needed to the final model. If the problems are unfixable, select a different candidate model.
      \end{enumerate}
    \end{frame}

    \begin{frame}{Backward stepwise regression}
      \begin{enumerate}
        \item Start with a ``full'' model containing all of the predictors.
        \item Remove the least significant (highest $p$-value / smallest $t$-statistic) predictor.
        \item Re-run the model with that predictor removed.
        \item Repeat steps 2-3 until all predictors are significant.
      \end{enumerate}
    \end{frame}

    \begin{frame}{Forward stepwise regression}
      \begin{enumerate}
        \item Start with a ``null'' model containing none of the predictors.
        \item Try adding each predictor, one at a time, and pick the one that ends up being the most significant (lowest $p$-value / highest $t$-statistic) predictor.
        \item Re-run the model with that predictor added.
        \item Repeat steps 2-3 until no more significant predictors can be added.
      \end{enumerate}
    \end{frame}
  \end{darkframes}

\end{document}
